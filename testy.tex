% !TEX encoding = UTF-8 Unicode
\documentclass{article}

\usepackage{polski}
\usepackage[utf8]{inputenc}
\usepackage{subfig}
\usepackage{graphicx}
\usepackage{hyperref}
\usepackage[a4paper, left=2.5cm, right=2.5cm, top=3.5cm, bottom=2.5cm, headsep=2.5cm]{geometry}

\linespread{1.3}
\begin{document}
	\section{Testy}

	Ponieważ projekt był tworzony w Visual Studio, to również testy wykonaliśmy w tym środowisku. Korzystaliśmy z Visual Studio Eksploratora testów.
	
	Najpierw należało utworzył nowy projekt, wybrać z listy szablonów „Projekt testu jednostki”, podać nazwę, a następnie dodać odwołanie z menu kontekstowego. W taki sposób powstała nowa klasa, która posłużyła nam do testów.
	
	Metoda testu musi spełniać następujące wymagania:
	\begin{itemize}
	\item 	Metodzie musi zostać nadany atrybut [TestMethod] .
	\item 	Metoda musi zwracać void.
	\item	Metoda nie może mieć parametrów.
	\end{itemize}
	
	Jeżeli w trakcie testów metoda zgłosi nam wyjątek inny, niż ten który podaliśmy, test kończy się wynikiem negatywnym.
	Ważne też jest pisanie testów. Testy jednostkowe powinny być atomowe, tzn. dotyczyć jednego przypadku użycia.
	Testy służą wyłapywaniu wyjątków. W naszym programie wyjątki mogły się pojawić, gdyby ktoś wpisał dane w niepoprawnym formacie.
	
	Pisząc testy musieliśmy zastanowić się, co użytkownik może wykonać w programie. Musieliśmy przewidzieć wszystkie możliwości, żeby program nie wyrzucił w niespodziewanym momencie wyjątku. Pisząc testy musieliśmy przemyśleć, co użytkownik może zrobić, ale też jaki skutek może mieć takie działanie.
	
\end{document}